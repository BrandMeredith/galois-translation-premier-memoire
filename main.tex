\documentclass{article}
\usepackage[utf8]{inputenc}

\title{Memoir on the Conditions of Solvability of Equations by Radicals}
\author{Évariste Galois}
\date{January 16, 1831}

\begin{document}

\maketitle

The attached memoir is extracted from a work that I had the honor of presenting to the Academy a year ago. That work not having been understood, the propositions it contained having been called into question, I have had to content myself with giving, in a synthesized way, the general principles and a \textit{single} application of my theory. I beg my judges to read at least these few pages carefully.

One will find herein a general \textit{condition} which \textit{each equation solvable by radicals satisfies}, and which reciprocally assures their solvability. We apply this only to equations for which the degree is a prime number. Here is the theorem given by our analysis:

\textit{
So that an equation of prime degree, which does not have commensurable divisors, should be solvable by radicals, it is necessary and sufficient that all the roots be rational functions of any two among them.
}
\\

The other applications of the theory are special theories in their own right. They require, however, the use of number theory and a special algorithm: we reserve them for another occasion. They are in part related to modular equations from the theory of elliptic functions, which we show not able to be solved by radicals.

\begin{flushright}
E. Galois \\
January 16, 1831
\end{flushright}


\section*{Principles}

We begin by establishing a few definitions and a series of lemmas, all of which are well-known.

\subsection*{Definitions}

An equation is said to be \textit{reducible} if it admits rational divisors; \textit{irreducible} otherwise.

It is necessary here to explain what is meant by the word "rational," as it will frequently appear.

When \textit{all} the coefficients of an equation are numerical and rational, this simply means that the equation can be decomposed into factors whose coefficients are numerical and rational.

However, when the coefficients of an equation are not all numerical and rational, a \textit{rational divisor} must be understood as a divisor whose coefficients can be expressed as rational functions of the proposed equation's coefficients.

>>>>>

Furthermore, we may agree to consider as rational any rational function of certain quantities that are determined and assumed to be known \textit{a priori}. For example, one may choose a certain root of an integer and consider any rational function of this radical as rational.

When we thus agree to consider certain quantities as known, we will say that we \textit{adjoin} these quantities to the equation being solved. We will say that these quantities are \textit{adjoined} to the equation.

\subsection*{Lemma I}
An irreducible equation cannot share any common root with a rational equation without dividing it.

\subsection*{Lemma II}
Given any equation without repeated roots, whose roots are $a, b, c, \ldots$, we can always construct a function $V$ of the roots such that no value obtained by permuting the roots in this function equals another.

\subsection*{Lemma III}
The function $V$, chosen as indicated above, will possess the property that all the roots of the proposed equation can be expressed rationally as functions of $V$.

\subsection*{Lemma IV}
Suppose we have formed the equation in $V$ and taken one of its irreducible factors so that $V$ is a root of an irreducible equation. Let $V, V', V'', \ldots$ be the roots of this irreducible equation. If $a = f(V)$ is one of the roots of the proposed equation, then $f(V')$ will similarly be a root of the proposed equation.

\end{document}


